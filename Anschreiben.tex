\documentclass{moderncv}
% Brief
\recipient{Company Recruitment team}{Company, Inc.\\Somestreet 123\\Somestadt}
\date{\today}
\opening{Sehr geehrte Damen und Herren,}
\closing{Mit freundlichen Grüßen}
\enclosure[Anhang]{Lebenslauf}

% Lebenslauf
\name{Max}{Mustermann}
\title{Zusammenfassender Titel}
\born{4. Juli 1776}
\address{Musterstr. 42}{53123 Bonn}{Germany}
\phone[mobile]{+1~(234)~567~890}
\phone[fixed]{+2~(345)~678~901}
\email{max@mustermann.org}
\homepage{https://www.maxmustermann.com}

\extrainfo{Zusätzliche Informationen}
\photo[64pt][0.4pt]{bilder/Profilfoto}
\quote{Irgendein Zitat}

\RequirePackage{aussehen/CV}

\begin{document}
    \section{Anschreiben}\label{sec:anschreiben}
    \makelettertitle

    Starte das Anschreiben mit einer kraftvollen Eröffnung.
    Stelle dich hier kurz vor und erläutere konkret, welche Stelle du anstrebst.
    Dein Ziel ist es, sofort das Interesse des Lesers zu wecken, deshalb könnte ein Hinweis auf eine gegenwärtige relevante Erfahrung oder eine kürzliche Errungenschaft sinnvoll sein.

    Im zweiten Absatz nimmst du Bezug auf das Unternehmen und die ausgeschriebene Stelle.
    Erkläre deine persönliche Motivation, warum du gerade für dieses Unternehmen und diese Position Interesse hast.
    Wichtig ist, dass du Bezug zu der Unternehmenskultur, den Unternehmenszielen oder aktuellen Projekten herstellst und klarmachst, wie deine Werte und beruflichen Ziele damit übereinstimmen.

    Der dritte Absatz dient dazu, deine Qualifikationen und Kompetenzen hervorzuheben.
    Setze dich mit den in der Stellenausschreibung genannten Anforderungen auseinander und lenke die Aufmerksamkeit auf deine Fähigkeiten, Erfahrungen und Erfolge, die dich für die Position passend machen.
    Verwende spezifische Beispiele und quantifiziere Errungenschaften, wenn möglich, um deine Angaben zu belegen.

    Bringe zum Schluss nochmals dein Interesse zum Ausdruck und danke dem Leser für die Zeit und die Berücksichtigung deiner Bewerbung.

    Ich freue mich auf die Gelegenheit, meine Eignung für die Position in einem persönlichen Gespräch weiter zu erläutern und danke Ihnen für die Berücksichtigung meiner Bewerbung.

    \makeletterclosing
\end{document}
