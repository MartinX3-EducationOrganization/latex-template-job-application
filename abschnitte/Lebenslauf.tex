%! Author = martin
%! Date = 29.06.21


\makecvtitle


\section{Bildung}\label{sec:bildung}
\tlcventry{2009}{2012}{Abschluss 1}{Institut}{Stadt}{\textit{Note}}{Beschreibung} % arguments 3 to 6 can be left empty
\tlcventry{2012}{2017}{Abschluss 2}{Institut}{Stadt}{\textit{Note}}{Beschreibung}


\section{Master Thesis}\label{sec:master-thesis}
\cvitem{Titel}{\emph{Titel}}
\cvitem{Betreuer}{Betreuer}
\cvitem{Beschreibung}{Kurze Zusammenfassung der Arbeit}


\section{Erfahrung}\label{sec:erfahrung}

\subsection{Beruflich}\label{subsec:beruflich}
\tlcventry{2017}{2021}{Jobtitel}{Arbeitgeber}{Stadt}{}{Generelle Beschreibung, die nicht länger als 1--2 Zeilen ist.\newline{}
Detaillierte Leistungen:
    \begin{itemize}
        \item Leistung 1
        \item Leistung 2 (with sub-leistungen)
        \begin{itemize}
            \item Sub-leistung (a);
            \item Sub-leistung (b), with sub-sub-leistungen (nicht machen!);
            \begin{itemize}
                \item Sub-sub-leistung i;
                \item Sub-sub-leistung ii;
                \item Sub-sub-leistung iii;
            \end{itemize}
            \item Sub-leistung (c);
        \end{itemize}
        \item Leistung 3
        \item Leistung 4
    \end{itemize}}
\tlcventry{2021}{2022}{Jobtitel}{Arbeitgeber}{Stadt}{}{Beschreibung Zeile 1\newline{}Beschreibung Zeile 2}

\subsection{Verschiedenes}\label{subsec:verschiedenes}
\tlcventry{2013}{2013}{Jobtitel}{Arbeitgeber}{Stadt}{}{Beschreibung}


\section{Sprachen}\label{sec:sprachen}
\cvitemwithcomment{Sprache 1}{Fähigkeitenniveau}{Kommentar}
\cvitemwithcomment{Sprache 2}{Fähigkeitenniveau}{Kommentar}
\cvitemwithcomment{Sprache 3}{Fähigkeitenniveau}{Kommentar}
\cvitemwithcomment{Sprache 4}{Fähigkeitenniveau}{Kommentar}


\section{IT-Erfahrungen}\label{sec:it-erfahrungen}
\cvdoubleitem{Kategorie 1}{XXX, YYY, ZZZ}{Kategorie 4}{XXX, YYY, ZZZ}
\cvdoubleitem{Kategorie 2}{XXX, YYY, ZZZ}{Kategorie 5}{XXX, YYY, ZZZ}
\cvdoubleitem{Kategorie 3}{XXX, YYY, ZZZ}{Kategorie 6}{XXX, YYY, ZZZ}


\section{Fähigkeitenmatrix}\label{sec:faehigkeitenmatrix}
\cvskilllegend[0.2em][Grundkenntnisse][Grundkenntnisse und eigene Erfahrung in Projekten][Umfangreiche Erfahrung in Projekten][Vertiefte Expertenkenntnisse][Experte\,/\,Spezialist]{Legende}
\cvskillhead[0em][Level][Fähigkeit][Jahre][Bemerkung]
\cvskillentry*{Sprache:}{3}{Python}{2}{Ich bin so erfahren in Python und habe schon eine Million Projekte realisiert. Mindestens.}
\cvskillentry{}{2}{LilyPond}{14}{So viele Notenblätter! Mann, ich bin der Beste!}
\cvskillentry{}{3}{\LaTeX}{14}{Offensichtlich rocke ich bei \LaTeX}
\cvskillentry*{OS:}{3}{Linux}{2}{Ich benutze übrigends nur Archlinux}
\cvskillentry*[1em]{Methoden}{4}{SCRUM}{8}{SCRUM Master seit 5 years}


\section{Interessen}\label{sec:Interessen}
\cvitem{Hobby 1}{Beschreibung}
\cvitem{Hobby 2}{Beschreibung}
\cvitem{Hobby 3}{Beschreibung}


\section{Extra 1}\label{sec:extra-1}
\cvlistitem{Element 1}
\cvlistitem{Element 2}
\cvlistitem{Element 3. Dieser Punkt ist besonders lang und erstreckt sich daher normalerweise über mehrere Zeilen. Haben Sie die Einrückung beim Zeilenumbruch bemerkt?}


\section{Extra 2}\label{sec:extra-2}
\cvlistdoubleitem{Element 1}{Element 4}
\cvlistdoubleitem{Element 2}{Element 5\cite{MaxMustermann2021a}}
\cvlistdoubleitem{Element 3}{Element 6. Wie Element 3 in der einspaltigen Liste zuvor ist dieses Element besonders lang, so dass es über mehrere Zeilen umgebrochen wird.}


\section{Referenzen}\label{sec:referenzen}
\begin{cvcolumns}
    \cvcolumn{Kategorie 1}{\begin{itemize}
                               \item Person 1\item Person 2\item Person 3
    \end{itemize}}
    \cvcolumn{Kategorie 2}{Unter anderem:\begin{itemize}
                                             \item Person 1, and\item Person 2
    \end{itemize}(Mehr auf Anfrage)}
    \cvcolumn[0.5]{Der ganze Rest \& einiges mehr}{\textit{Diese} Person, und auch \textbf{solche} (Alles auf Anfrage verfügbar).}
\end{cvcolumns}~\nocite{*} % Zum Auflisten aller Einträge, auch wenn sie nicht zitiert werden.
\printbibliography[title={Veröffentlichungen}]
